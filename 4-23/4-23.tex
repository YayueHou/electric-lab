\documentclass{article}
\usepackage{graphicx}
\usepackage{listings}
\usepackage{ctex}
\usepackage{graphicx}
\usepackage[a4paper, body={18cm,22cm}]{geometry}
\usepackage{amsmath,amssymb,amstext,wasysym,enumerate,graphicx}
\usepackage{float,abstract,booktabs,indentfirst,amsmath}
\usepackage{array}
\usepackage{booktabs} %调整表格线与上下内容的间隔
\usepackage{multirow}
\usepackage{diagbox}
\usepackage{indentfirst}
\usepackage{bm}
\usepackage{fancyhdr}




\pagestyle{fancy}

\lhead{\bfseries \normalsize 学号:1952033\quad 姓名:侯雅玥 \quad 组员:廖宏 \\实验名称:集成移位寄存器应用实验\quad 课程名称:电子技术实验\quad 专业:微电子科学与工程 } 
\rhead{}

\begin{document}
	\section{\zihao{4} 实验名称:集成移位寄存器应用实验}
    \section{\zihao{4} 实验目的}
    \zihao {5} (1)熟悉 555 定时器电路结构、工作原理及特点。\par 
               (2)掌握 555 定时器的基本应用。\par
               (3)熟悉用示波器测量 555 定时器的脉冲幅度、周期和脉冲宽度。 \par
   	\section{\zihao{4} 实验原理}
      
       55定时器是由模拟和数字电路相混合构成的集成电路。由于电路中使用了三个 5kΩ的电阻,故取名为 555电路。
       555电路只要外接少量阻容元器件,就可以组成单稳态触发器、多谐振荡器、多种波形发生器等。
       由于电路结构简单,性能可靠,使用方便,故应用范围很广泛。555电路的内部结构框和引脚排列图如图1、图2所示。
       \begin{figure}[h]
        \begin{minipage}[t]{0.5\linewidth} % 如果一行放2个图,用0.5,如果3个图,用0.33  
          \centering   
          \includegraphics[width=3.5in]{H:/电子技术试验/4-23/4-23-1.png}   
          \caption{555定时器内部结构图}   
          \label{fig:side:a}   
        \end{minipage}%   
        \begin{minipage}[t]{0.5\linewidth}   
          \centering   
          \includegraphics[width=3.5in]{H:/电子技术试验/4-23/4-23-2.png}   
          \caption{555定时器引脚图}   
          \label{fig:side:b}   
        \end{minipage}   
      \end{figure}

      555 定时器主要由两个电压比较器($C_1$和$C_2$)、一个基本 RS触发器、一个泄放三极管 $T_D$和三个5kΩ电阻构成的分压器组成。
      与非门$G_1$和$G_2$,构成基本 RS 触发器,输入$R_D$'为复位端,低电平有效。比较器$C_1$和 $C_2$的输出$U_{C1}$和$U_{C2}$为
       RS触发器的触发信号。若比较器 $C_1(C_2)$的"+"输入端电位低于"一"输入端电位,即 $U_+<U_-$则输出U+为低电平(Uc="0"),
       反之输出Uc为高电平(Uc="1")。比较器C1的参考电压$U_{1+} =\frac{2}{3} U_{CC}$,
      比较器 C2的参考电压 $U_{2_-}=\frac{1}{3}U_{CC}$,如果$U_{1+}$的
      外接端CO接固定电压$U_{CO}$,则$U_{1+} =U_{CO}$,$U_{2_-}=\frac{1}{2}U_{CO}$泄放三极管$T_D$为外接电容提供充、放电回路。
      反相器$G_3$为输出缓冲反相器,起整形和提高带负载能力的作用。\par
      从图2的555定时器引脚排列图可以看出,引脚4为复位端$R_D$'引脚5为电压控制端CO,可以改变比较器$C_1$和$C_2$的上、
      下参考电位$U_{1+}$和$U_{2-}$引脚2为低电平触发端TR;引脚6为高电平触发端 TH;引脚7为$T_D$的集电极开路输出(放电)端DISC。\par
(1)555 定时器构成单稳电路\par
图3为 555 定时器构成的单稳态触发器电路,复位端 $R_D$'接高电平 $U_{CC}$。触发信号ui从低电平触发端TR'输入,所以电路在ui的下降沿触发。
三极管 $T_D$的集电极输出 DISC端通过电阻R 接$U_{CC}$,构成反相器。反相器的输出(DISC)同时接电容 C,555定时器的高电平触发端 TH也与DISC 端相连,
从而构成积分型单稳态触发器。其工作波形如图4所示。\par

\begin{figure}[h]
    \begin{minipage}[t]{0.5\linewidth} % 如果一行放2个图,用0.5,如果3个图,用0.33  
      \centering   
      \includegraphics[width=3.5in]{H:/电子技术试验/4-23/3.png}   
      \caption{555定时器内部结构图}   
      \label{fig:side:a}   
    \end{minipage}%   
    \begin{minipage}[t]{0.5\linewidth}   
      \centering   
      \includegraphics[width=3.5in]{H:/电子技术试验/4-23/2.jpg}   
      \caption{555定时器引脚图}   
      \label{fig:side:b}   
    \end{minipage}   
  \end{figure}


当 CO端不外接控制电压时,该单稳态触发器的输出脉冲宽度$ t_w$为
\[   t_w=RCln\frac{U_{CC}{U_{CC}-\frac{2}{3}U_{CC}}\approx 1.1RC  \]
\par
$t_w$由定时元件R与C参数决定,改变 R 与C 值,可以控制输出波形的宽度。因此,单稳态触发器常用于定时、延迟或整形电路。\par
(2)用 555 定时器构成多谐振荡器\par
图5是 555定时器构成多振荡器电路,高、低电平触发输入 TH与TR'相连作为输入。电压控制端CO接0.01μF 电容滤波,$R_D$端接高电平。
三极管$ T_D$集电极上拉电阻 $R_A$至电源$U_{CC}$构成反相器,反相器输出DISC通过$R_BC$积分电路反馈至输人 TH和TR',组成自激多谐热荡器。
此电路没有稳态,也不需外加触发信号,电源通过$R_A$和 $R_B$向C充电以及C通过$R_B$间DISC端放电,使电路自动在两个暂稳态之间变化,
形成振荡信号输出、可以分析,在容充电时,电路的暂稳态持续时间为
\[ t_{w1}=0.7(R_A+R_B)C\]
在电容 C放电时,暂稳态持续时间为
\[t_{w2}=0.7R_BC\]
因此,电路输出矩形脉冲信号的周期为
\[T=t_{w1}+t_{w2}=0.7(R_A+2R_B)C\]
输出矩形脉冲的占空比为
\[q=\frac{t_{w1}}{T}=\frac{R_A+R_B}{R_A+2R_B}\]

可见,通过改变电阻 $R_A$与 $R_B$和电容 C的参数,即可改变振荡信号频率。振荡信号的占空比由$R_A$与 $R_B$的参数决定,但无法小于 
50\%。要使多谐振荡器的占空比在 50\%以下的范围可调,必须使电容的充、放回路互相独立。
那么可以在图5所示电路上增加 1个电位器和2 个二极管来实现,如图 6所示。

\begin{figure}[h]
    \begin{minipage}[t]{0.5\linewidth} % 如果一行放2个图,用0.5,如果3个图,用0.33  
      \centering   
      \includegraphics[width=3.5in]{H:/电子技术试验/4-23/5.png}   
      \caption{555定时器内部结构图}   
      \label{fig:side:a}   
    \end{minipage}%   
    \begin{minipage}[t]{0.5\linewidth}   
      \centering   
      \includegraphics[width=3.5in]{H:/电子技术试验/4-23/6.png}   
      \caption{555定时器引脚图}   
      \label{fig:side:b}   
    \end{minipage}   
  \end{figure}


\section{\zihao{4} 实验内容及步骤}
1.构成单稳态触发器:按图3接线,C=0.1µF,输入端加1kHz的脉冲信号,用示波器观察Ui,Uc,Uo的波形。改变电位器Rw的阻值,测量输出脉冲宽度tw的变化范围,并与理论值相比较。\par
2.构成多谐振荡器:按图5连接电路,其中,RA为10kΩ电阻,RB由100kΩ电位器和10kΩ电阻串联构成,电容C为0.01µF。调节RB使输出Uo的频率f=1kHz,记录此时的Uo,Uc的波形和RB的实际阻值。\par
3.占空比可调的脉冲信号发生器:按图6连接电路,其中R1=R2=10kΩ,Rw=100kΩ。改变电位器Rw值,组成一个占空比为50\%的脉冲信号发生器,用示波器记录Uo,Uc的波形。Rw变化时,记录占空比的变化范围。(C=0.01µF)\par


\section{\zihao{4} 实验设备和器材}
(1)直流稳压电源              \qquad \quad \qquad \qquad \qquad \qquad           1台\par
(2)数字逻辑实验箱            \qquad  \qquad \qquad \qquad\qquad                1台\par
(3)74LS00、74LS161、74LS194              \quad                                    若干片\par
(4)示波器                   \qquad  \qquad \qquad \qquad\qquad  \qquad  \qquad    1台\par
(5)导线   
\section{\zihao{4} 数据处理}
(1)555单稳态触发器\par
$u_i,u_c$波形
\begin{figure}[h]
    \begin{minipage}[t]{0.5\linewidth} % 如果一行放2个图,用0.5,如果3个图,用0.33  
      \centering   
      \includegraphics[width=3.5in]{H:/电子技术试验/4-23/SCR28.png}   
      \caption{$R_w=116.2k\Omega$}   
      \label{fig:side:a}   
    \end{minipage}%   
    \begin{minipage}[t]{0.5\linewidth}   
      \centering   
      \includegraphics[width=3.5in]{H:/电子技术试验/4-23/SCR31.png}   
      \caption{$R_w=0k\Omega$}   
      \label{fig:side:b}   
    \end{minipage}   
  \end{figure}
  $u_i,u_0$波形
  \begin{figure}[h]
    \begin{minipage}[t]{0.5\linewidth} % 如果一行放2个图,用0.5,如果3个图,用0.33  
      \centering   
      \includegraphics[width=3.5in]{H:/电子技术试验/4-23/SCR34.png}   
      \caption{$R_w=116.2k\Omega$}   
      \label{fig:side:a}   
    \end{minipage}%   
    \begin{minipage}[t]{0.5\linewidth}   
      \centering   
      \includegraphics[width=3.5in]{H:/电子技术试验/4-23/SCR33.png}   
      \caption{$R_w=0k\Omega$}   
      \label{fig:side:b}   
    \end{minipage}   
  \end{figure}
\par
  (2)多谐振荡器\par
  $u_i,u_c$波形
  \begin{figure}[h]
        \centering   
        \includegraphics[width=3.5in]{H:/电子技术试验/4-23/SCR35.png}   
        \caption{$U_0,U_c$}   
        \label{fig:side:a}   
    \end{figure}
\par
此时测得$R_B=41.3k\Omega$
\par
(3)占空比可调的脉冲信号发生器\par
  $u_0,u_c$波形
  \begin{figure}[h]
        \centering   
        \includegraphics[width=3.5in]{H:/电子技术试验/4-23/SCR36.png}   
        \caption{占空比50\%}   
        \label{fig:side:a}   
    \end{figure}

    调节$R_W$
    \begin{figure}[h]
        \begin{minipage}[t]{0.5\linewidth} % 如果一行放2个图,用0.5,如果3个图,用0.33  
          \centering   
          \includegraphics[width=3.5in]{H:/电子技术试验/4-23/SCR38.png}   
          \caption{$R_w=116.2k\Omega$}   
          \label{fig:side:a}   
        \end{minipage}%   
        \begin{minipage}[t]{0.5\linewidth}   
          \centering   
          \includegraphics[width=3.5in]{H:/电子技术试验/4-23/SCR37.png}   
          \caption{$R_w=0k\Omega$}   
          \label{fig:side:b}   
        \end{minipage}   
      \end{figure}
 \section{\zihao{4} 误差处理}
(1)单稳态触发器\par 
\[ t_{w01}=1.1RC=13.90ms \]
\[ t_{w02}=1.1RC=1.1ms \]
\[ \Delta t_0=12.8ms \]
\[ \delta (t_{w1)}=\frac{ t_{w1}- t_{w01}}{t_{w1}}=15\%\]
\[ \delta (t_{w2})=\frac{ t_{w2}- t_{w02}}{t_{w2}}=45\%\]
\[ \delta (\Delta t)=\frac{ \Delta t- \Delta t_0}{\Delta t}=11\%\]
(2)多谐振荡器\par
\[q=\frac{t_{w1}}{T}=\frac{R_A+R_B}{R_A+2R_B}=55.4\%\]
\[\delta (q)=\frac{q-q_0}{q_0}=1\%\]

\par
单稳态触发器的$t_w$误差较大可能是芯片质量导致的。\par
多谐振荡器误差合理。\par
\section{结论}
(1)单稳电路的$t_w$由定时元件R与C参数决定,改变 R 与C 值,可以控制输出波形的宽度。\par
(2)多谐振荡器,通过改变电阻 $R_A$与 $R_B$和电容 C的参数,即可改变振荡信号频率。振荡信号的占空比由$R_A$与 $R_B$的参数决定,但无法小于 
50\%。\par
(3)在多谐振荡器电路上增加 1个电位器和2 个二极管可以实现调节占空比\par
\section{思考题}
(1)555定时器构成的单稳态触发器的脉冲宽度和周期由什么决定?R、C取值应怎样分配?为什么?\par
由外接RC的数值决定,C值不能过大,导致电路不稳定。
(2)若单稳态触发器的输入脉宽$t_i$大于$t_w$,时希望电路能正常工作,电路该怎样改进?\par 
可以先通过RC微分电路把输入脉宽变窄,然后再输入555定时器的2脚上。或改变$R_w$的最大阻值,或增大C的电容值,使$t_w$增大到大于$t_i$的程度。
(3)555定时器构成的多谐振荡器,其振荡周期和占空比的改变与哪些因素关?\par 
由于有
\[T=t_{w1}+t_{w2}=0.7(R_A+2R_B)C\]
\[q=\frac{t_{w1}}{T}=\frac{R_A+R_B}{R_A+2R_B}\]
因此周期和占空比与$R_A,R_B$的值有关\par 
(4)如何用 555 定时器构成施密特触发器?
将芯片的6脚和2脚相连,作为输入端Vi,由3脚或7脚挂接电阻RL及电源$V_{DD}$作为输出端,即能构成施密特触发器。
\begin{figure}[h]
  \centering   
  \includegraphics[width=3.5in]{H:/电子技术试验/4-23/7.png}   
  \caption{占空比50\%}   
  \label{fig:side:a}   
\end{figure}

\end{document}